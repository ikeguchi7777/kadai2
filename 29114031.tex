\documentclass{jarticle}

\usepackage{graphicx}
\usepackage{url}
\usepackage{listings,jlisting}
\usepackage{ascmac}
\usepackage{amsmath,amssymb}

%ここからソースコードの表示に関する設定
\lstset{
  basicstyle={\ttfamily},
  identifierstyle={\small},
  commentstyle={\smallitshape},
  keywordstyle={\small\bfseries},
  ndkeywordstyle={\small},
  stringstyle={\small\ttfamily},
  frame={tb},
  breaklines=true,
  columns=[l]{fullflexible},
  numbers=left,
  xrightmargin=0zw,
  xleftmargin=3zw,
  numberstyle={\scriptsize},
  stepnumber=1,
  numbersep=1zw,
  lineskip=-0.5ex
}
%ここまでソースコードの表示に関する設定 

\title{知能プログラミング演習II 課題2}
\author{グループ07\\
  29114031 大原 拓人\\
%  {\small (グループレポートの場合は、グループ名および全員の学生番号と氏名が必要)}
}
\date{2019年10月28日}

\begin{document}
\maketitle

\paragraph{提出物} 個人レポート、グループプログラム「group07.zip」
\paragraph{グループ} グループ07
\paragraph{メンバー}
\begin{tabular}{|c|c|c|}
  \hline
  学生番号&氏名&貢献度比率\\
  \hline\hline
  29114007&池口弘尚&100\\
  \hline
  29114031&大原拓人&100\\
  \hline
  29114048&北原太一&100\\
  \hline
  29114086&飛世裕貴&100\\
  \hline
  29114095&野竹浩二朗&100\\
  \hline
\end{tabular}

\section{課題の説明}
\begin{description}
    \item[必須課題2-1] MatchingクラスまたはUnifyクラスを用い,パターンで検索可能な簡単なデータベースを作成せよ.
    与えられたパターンにマッチする全データを列挙するプログラムを作成せよ.
    \\ 例えば,この例のような形式のデータセットから,?x has a hobby of playing video-games や 
    Hanako is a ?y のような,様々なパターンにマッチするデータを検索できるようにすること.
    複数のパターンが与えられたときに全ての可能な変数束縛の集合を返すようなプログラムを作成せよ.
    \\ 例えば,上記の例で「?x is a boy」と「?x loves ?y」の両方が与えられたときに,(?x, ?y) の
    全ての可能な変数束縛の集合として{(Taro, Jiro), (Jiro, Hanako)}を返すこと.
    \item[必須課題2-2] 自分たちの興味ある分野の知識についてデータセットを作り,上記2-1で実装したデータベースに登録せよ.
    また,検索実行例を示せ.どのような方法でデータセットを登録しても構わない.
    \item[発展課題2-3] 上記システムのGUIを作成せよ.
    \\ データの追加,検索,削除をGUIで操作できるようにすること.
    \\ 登録されたデータが次回起動時に消えないよう,登録されたデータをファイルへ書き込んだり読み込んだりできるようにすること.
\end{description}


\section{課題2-1}
\begin{screen}
    MatchingクラスまたはUnifyクラスを用い,パターンで検索可能な簡単なデータベースを作成せよ.
    与えられたパターンにマッチする全データを列挙するプログラムを作成せよ.
    \\ 例えば,この例のような形式のデータセットから,?x has a hobby of playing video-games や 
    Hanako is a ?y のような,様々なパターンにマッチするデータを検索できるようにすること.
    複数のパターンが与えられたときに全ての可能な変数束縛の集合を返すようなプログラムを作成せよ.
    \\ 例えば,上記の例で「?x is a boy」と「?x loves ?y」の両方が与えられたときに,(?x, ?y) の
    全ての可能な変数束縛の集合として{(Taro, Jiro), (Jiro, Hanako)}を返すこと.
\end{screen}
\subsection{手法}
    配列にデータベース用の文字列を格納し、for文を用いて配列の中を走査するようにした。

\subsection{実装}

もともと与えられたSearch.javaを以下のように変更した。

\begin{lstlisting}[caption=クラスより抜粋]
\end{lstlisting}

\begin{lstlisting}[caption=結果を保存し、比較]
\end{lstlisting}

\subsection{実行例}

\subsection{考察}

\section{課題2-2}
\begin{screen}
    自分たちの興味ある分野の知識についてデータセットを作り,上記2-1で実装したデータベースに登録せよ.
    また,検索実行例を示せ.どのような方法でデータセットを登録しても構わない.
\end{screen}

\subsection{手法}

\subsection{実装}

\subsection{実行例}

\subsection{考察}

\section{課題2-3}
\begin{screen}
    課題2-1で作成したシステムのGUIを作成せよ.
    \\ データの追加,検索,削除をGUIで操作できるようにすること.
    \\ 登録されたデータが次回起動時に消えないよう,登録されたデータをファイルへ書き込んだり読み込んだりできるようにすること.
\end{screen}
\subsection{手法}

\subsection{実装}

\subsection{実行例}

\subsection{考察}

\section{感想}
\subsection{}

% 参考文献
\begin{thebibliography}{99}
  \bibitem{text} 知能処理学の講義スライド、主に分枝限定法の部分
  \bibitem{goo} 矢印を描画 -JAVAで矢印を描画したいのですが、どうしたらいいのかわか- Java | 教えて!goo
  \url{https://oshiete.goo.ne.jp/qa/4014364.html} (2019年10月7日アクセス).
  \bibitem{swing} Swingを使ってみよう - Java GUIプログラミング
  \url{https://www.javadrive.jp/tutorial/} (2019年10月7日アクセス).
\end{thebibliography}

\end{document}