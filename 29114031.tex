\documentclass{jarticle}

\usepackage{graphicx}
\usepackage{url}
\usepackage{listings,jlisting}
\usepackage{ascmac}
\usepackage{amsmath,amssymb}

%ここからソースコードの表示に関する設定
\lstset{
    basicstyle={\ttfamily},
    identifierstyle={\small},
    commentstyle={\smallitshape},
    keywordstyle={\small\bfseries},
    ndkeywordstyle={\small},
    stringstyle={\small\ttfamily},
    frame={tb},
    breaklines=true,
    columns=[l]{fullflexible},
    numbers=left,
    xrightmargin=0zw,
    xleftmargin=3zw,
    numberstyle={\scriptsize},
    stepnumber=1,
    numbersep=1zw,
    lineskip=-0.5ex
}
%ここまでソースコードの表示に関する設定 

\title{知能プログラミング演習II 課題2}
\author{グループ07\\
    29114031 大原 拓人\\
%  {\small (グループレポートの場合は、グループ名および全員の学生番号と氏名が必要)}
}
\date{2019年10月28日}

\begin{document}
\maketitle

\paragraph{提出物} 個人レポート、グループプログラム「group07.zip」
\paragraph{グループ} グループ07
\paragraph{メンバー}
\begin{tabular}{|c|c|c|}
    \hline
    学生番号&氏名&貢献度比率\\
    \hline\hline
    29114007&池口弘尚&100\\
    \hline
    29114031&大原拓人&100\\
    \hline
    29114048&北原太一&100\\
    \hline
    29114086&飛世裕貴&100\\
    \hline
    29114095&野竹浩二朗&100\\
    \hline
\end{tabular}

\section{課題の説明}
\begin{description}
    \item[必須課題2-1] MatchingクラスまたはUnifyクラスを用い,パターンで検索可能な簡単なデータベースを作成せよ.
    与えられたパターンにマッチする全データを列挙するプログラムを作成せよ.
    \\ 例えば,この例のような形式のデータセットから,?x has a hobby of playing video-games や 
    Hanako is a ?y のような,様々なパターンにマッチするデータを検索できるようにすること.
    複数のパターンが与えられたときに全ての可能な変数束縛の集合を返すようなプログラムを作成せよ.
    \\ 例えば,上記の例で「?x is a boy」と「?x loves ?y」の両方が与えられたときに,(?x, ?y) の
    全ての可能な変数束縛の集合として{(Taro, Jiro), (Jiro, Hanako)}を返すこと.
    \item[必須課題2-2] 自分たちの興味ある分野の知識についてデータセットを作り,上記2-1で実装したデータベースに登録せよ.
    また,検索実行例を示せ.どのような方法でデータセットを登録しても構わない.
    \item[発展課題2-3] 上記システムのGUIを作成せよ.
    \\ データの追加,検索,削除をGUIで操作できるようにすること.
    \\ 登録されたデータが次回起動時に消えないよう,登録されたデータをファイルへ書き込んだり読み込んだりできるようにすること.
\end{description}


\section{課題2-1}
\begin{screen}
    MatchingクラスまたはUnifyクラスを用い,パターンで検索可能な簡単なデータベースを作成せよ.
    与えられたパターンにマッチする全データを列挙するプログラムを作成せよ.
    \\ 例えば,この例のような形式のデータセットから,?x has a hobby of playing video-games や 
    Hanako is a ?y のような,様々なパターンにマッチするデータを検索できるようにすること.
    複数のパターンが与えられたときに全ての可能な変数束縛の集合を返すようなプログラムを作成せよ.
    \\ 例えば,上記の例で「?x is a boy」と「?x loves ?y」の両方が与えられたときに,(?x, ?y) の
    全ての可能な変数束縛の集合として{(Taro, Jiro), (Jiro, Hanako)}を返すこと.
\end{screen}
\subsection{手法}
    配列にデータベース用の文字列を格納し、for文を用いて配列の中を走査するようにした。
    統合開発環境を用いているため、コマンドライン引数の変更の手間が省けるように、
    入力ストリームから検索するパターンを読み込むように変更した。
    検索の効率をよくするためにDataBaseクラスのSearchメソッドで絞り込みを行ってから、走査するようにした。
    複数のパターンを読み込めるようにし、変数束縛の集合と一致したパターンの数を引数で渡しつつ、
    再帰アルゴリズムを用いて検索するsearchメソッドを作成した。

\subsection{実装}
    もともと与えられたUnify.javaを以下のように変更した。
    \paragraph{}
    まず、コマンドライン引数で使用するデータセットを読み込む。
    読み込みが完了したら、検索するパターンを入力するように促して
    入力ストリームからパターンを読み取る。読み取ったパターンは、
    Unifierインスタンスを作成し、そのsearchメソッドの引数として渡す。

\begin{lstlisting}[caption=Unifyクラスより抜粋]
public static void main(String arg[]) {
// For CUI Test
Scanner stdin = new Scanner(System.in);
    for (int i = 0; i < arg.length; i++) {
        try {
            BufferedReader reader = new BufferedReader(new InputStreamReader(new FileInputStream(arg[i]), "UTF-8"));
            String s = reader.readLine();
            while (s != null) {
                if (!s.equals(""))
                    DataBase.getDataBase().insert(s);
                s = reader.readLine();
            }
            reader.close();
        } catch (UnsupportedEncodingException e) {
            // TODO Auto-generated catch block
            e.printStackTrace();
        } catch (FileNotFoundException e) {
            // TODO Auto-generated catch block
            e.printStackTrace();
        } catch (IOException e) {
            // TODO Auto-generated catch block
            e.printStackTrace();
        }
    }
    while (true) {
        System.out.println("Enter Search Pattern:");
        String scan = stdin.nextLine();
        if (scan.equals("exit"))
            break;
        (new Unifier()).search(scan);
        for (String string : Unifier.getMatchList())
            System.out.println(string);
        for (String string : Unifier.getVarSets())
            System.out.println(string);
    }
    stdin.close();
}
\end{lstlisting}
    \paragraph{}
    絞り込みで得られたデータを使い回すためにstatic変数としている。
    また、マッチングに成功したときのデータと変数束縛の集合も
    探索後に消えてしまわないように、static型に格納して後から
    アクセスできるようにした。マッチングを始める前に、
    これらのstatic変数は初期化される。
\begin{lstlisting}[caption=Unifierクラスの主要パーツ]
class Unifier {
    private static DataBase[] results;
    private static String[] terms;
    private static ArrayList<String> varSets;
    private static ArrayList<String> matchList;
    ...
    //ここでそのほかの変数、コンストラクタ、セッタ、ゲッタなどを定義
    ...

    public static void setByTerms(String str) {
        // 検索するパターンをもとにDataBaseクラスのSearchメソッドで絞り込み格納
        String[] terms = str.split(";");
        Unifier.terms = terms;
        DataBase[] results = new DataBase[terms.length];
        for (int i = 0; i < results.length; i++) {
            results[i] = DataBase.getDataBase().Search(terms[i]);
        }
        Unifier.results = results;
    }

    private static void addMatchList(ArrayList<String> matched) {
        // 重複するものは加えない
        for (String addline : matched) {
            boolean match = false;
            for (String string : Unifier.matchList) {
                if (addline.equals(string)) {
                    match = true;
                    break;
                }
            }
            if (!match)
                Unifier.matchList.add(addline);
        }
    }

    public boolean search(String str) {
        // 結果の初期化
        cleanVarSets();
        cleanMatchList();
        setByTerms(str);
        return search(new HashMap<>(), new ArrayList<>(), 0);
    }
    ...
\end{lstlisting}
    \paragraph{}
    それまでに得られた変数束縛の組と、マッチしたパターンの組、何個目の式の
    マッチングかを表す整数を引数として、再帰アルゴリズムを用いて実装した。
    一つでもマッチする組み合わせがあれば、trueを返す。
\begin{lstlisting}[caption=searchメソッド]
...
boolean search(HashMap<String, String> vars, ArrayList<String> matched, int layer) {
    // 再帰でバックトラック可能に
    boolean match = false;
    if (layer < terms.length) {
        if (Unifier.results[layer] == null)
            return false;
        Unifier unifier = new Unifier(Unifier.results[layer].GetResult());
        for (String string : unifier.lines) {
            unifier.setVars(vars);
            unifier.setMatched(matched);
            if (unifier.unify(string, terms[layer])) {
                unifier.matched.add(string);
                match = search(unifier.vars, unifier.matched, layer + 1) || match;
            }
        }
    } else {
        // すべての層(パターン式)を通過すれば結果に追加
        addMatchList(matched);
        String varset = vars.toString();
        for (String string : matched)
            // マッチした文を結果に追加
            addVarSets(string);
        if (varset.equals("{}"))
            // 変数がなければtrue.
            addVarSets("true.");
        else
            addVarSets(varset);
        return true;
    }
    return match;
}
...
\end{lstlisting}

\subsection{実行例}
ここではmoodleで与えられたデータセットの例を用いた。
\begin{lstlisting}[caption=Unify.javaの実行例]]
Enter Search Pattern:
Taro is a boy
Taro is a boy
true.
Enter Search Pattern:
?x is a boy  
Taro is a boy
{?x=Taro}
Jiro is a boy
{?x=Jiro}
Enter Search Pattern:
?x loves ?y;?y loves ?z;?z loves ?x
Hanako loves Taro
Taro loves Jiro
Jiro loves Hanako
{?x=Hanako, ?y=Taro, ?z=Jiro}
Taro loves Jiro
Jiro loves Hanako
Hanako loves Taro
{?x=Taro, ?y=Jiro, ?z=Hanako}
Jiro loves Hanako
Hanako loves Taro
Taro loves Jiro
{?x=Jiro, ?y=Hanako, ?z=Taro}
Enter Search Pattern:
exit
\end{lstlisting}
    \paragraph{}
    変数がなければ、「true.」が表示される。
    複数のパターンをセミコロン「;」で区切って入力することでそれらの検索ができ、
    一致したパターンとその時の変数束縛の集合が表示されている。

\subsection{考察}
もともと入力されたパターンごとにデータセットを全走査する予定だったが、
グループの池口君が絞り込みにかなり有効なDataBaseクラスを作ってくれたので、
探索の効率がかなり向上した。変数束縛の組を共有せず、インスタンスごとに
複製して引き継ぐことで、バックトラックらしい挙動を再現できた。

\section{課題2-2}
\begin{screen}
    自分たちの興味ある分野の知識についてデータセットを作り,上記2-1で実装したデータベースに登録せよ.
    また,検索実行例を示せ.どのような方法でデータセットを登録しても構わない.
\end{screen}
\subsection{手法}

\subsection{考察}
この課題はグループのほかの人に託したので、グループレポートを参照されたい。

\section{課題2-3}
\begin{screen}
    課題2-1で作成したシステムのGUIを作成せよ.
    \\ データの追加,検索,削除をGUIで操作できるようにすること.
    \\ 登録されたデータが次回起動時に消えないよう,登録されたデータをファイルへ書き込んだり読み込んだりできるようにすること.
\end{screen}
\subsection{手法}

\subsection{実装}
私は、UnifierクラスからGUI用に出力できるように、
リストに結果を格納するように調整した。

\subsection{考察}
削除するためにマッチしたパターンを重複なく格納するリストが必要になったので、
課題2-1で示したマッチしたパターンをパターン発見時に保存する機能は、GUI用に後から実装した。

\section{感想}
時間が足りなくて、構造化したパターンを解析するプログラムまではできなかった。
はじめは複数パターンの照合を多重のforループで書こうとしていたが、ループの層の数がパターンの
数によって変わるので、再帰アルゴリズムが必要だと思い付いた。
再帰アルゴリズムのためには、メソッドのコールごとに変数を保持して、
そのコピーを引数で渡していくのだが、当初はデータベースの絞り込んだ結果の
配列のポインタも引数として渡そうとしていたので、読みづらいコードに
なっていたが、クラスのstatic変数を用いることで引数に含めなくてもよくなった。
あらためて、クラスの使い方を確認できた。

% 参考文献
\begin{thebibliography}{99}
    \bibitem{pl} prologのプログラムや挙動を参考にした
\end{thebibliography}

\end{document}