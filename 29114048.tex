\documentclass{jarticle}

\usepackage{graphicx}
\usepackage{url}
\usepackage{listings,jlisting}
\usepackage{ascmac}
\usepackage{amsmath,amssymb}

%ここからソースコードの表示に関する設定
\lstset{
  basicstyle={\ttfamily},
  identifierstyle={\small},
  commentstyle={\smallitshape},
  keywordstyle={\small\bfseries},
  ndkeywordstyle={\small},
  stringstyle={\small\ttfamily},
  frame={tb},
  breaklines=true,
  columns=[l]{fullflexible},
  numbers=left,
  xrightmargin=0zw,
  xleftmargin=3zw,
  numberstyle={\scriptsize},
  stepnumber=1,
  numbersep=1zw,
  lineskip=-0.5ex
}
%ここまでソースコードの表示に関する設定

\title{知能プログラミング演習II 課題1}
\author{グループ7\\
  29114048 北原 太一\\
%  {\small (グループレポートの場合は、グループ名および全員の学生番号と氏名が必要)}
}
\date{2019年10月29日}

\begin{document}
\maketitle

\paragraph{提出物} rep2
\paragraph{グループ} グループ7
\paragraph{メンバー}
\begin{tabular}{|c|c|c|}
  \hline
  学生番号&氏名&貢献度比率\\
  \hline\hline
  29114007&池口弘尚&100\\
  \hline
  29114031&大原拓人&100\\
  \hline
  29114048&北原太一&100\\
  \hline
  29114086&飛世裕貴&100\\
  \hline
  29114095&野竹浩二朗&100\\
  \hline
\end{tabular}

\section{課題の説明}
\begin{description}
\item[課題2-1] MatchingクラスまたはUnifyクラスを用い,パターンで検索可能な簡単なデータベースを作成せよ.
\item[課題2-2] 自分たちの興味ある分野の知識についてデータセットを作り,上記2-1で実装したデータベースに登録せよ.また,検索実行例を示せ.どのような方法でデータセットを登録しても構わない.
\item[課題2-3] 上記システムのGUIを作成せよ.
\end{description}


\section{課題2-2}
\begin{screen}
  自分たちの興味ある分野の知識についてデータセットを作り,上記2-1で実装したデータベースに登録せよ.また,検索実行例を示せ.どのような方法でデータセットを登録しても構わない.
\end{screen}

私の担当箇所は、データセットの作成である。

\subsection{実装}
データセットは、3名で作成したが、監視の有無や使う単語などの統一を図った。具体的な内容については、Dataset.txtを参照されたい。

\subsection{考察}
今回は1人が作成したcsvライクなファイルからデータセットを手動で生成したが、何かしらのプログラムにより自動化すれば効率的に文を生成でき、残りの課題の高クォリティ化に繋がったかもしれない。


\section{課題2-3}
\begin{screen}
  上記システムのGUIを作成せよ.
\end{screen}

私の担当箇所は、デバッグ及び考察である。したがって、実行例及び考察のみを記す。

\subsection{実行例}

Moodleのデータセットサンプルの、上から4行を登録し、そのまま保存した。その際に保存されたファイルの中身を以下に記す。

\begin{lstlisting}
student is a kind of human
Hanako is a girl
Hanako is a student
human is a kind of mammal
\end{lstlisting}

\subsection{考察}
  登録した内容が、過不足なく登録されている。このファイルから登録すれば、以前に登録した内容をそのまま引き継げるため、データベースの内容を保存することができていると言える。

\end{document}
